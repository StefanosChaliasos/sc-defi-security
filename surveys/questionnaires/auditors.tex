\documentclass[a4paper,10pt]{article}
\usepackage[ngerman]{babel}
\usepackage[utf8]{inputenc}
\usepackage{wasysym}% provides \ocircle and \Box
\usepackage{enumitem}% easy control of topsep and leftmargin for lists
\usepackage{color}% used for background color
\usepackage{forloop}% used for \Qrating and \Qlines
\usepackage{ifthen}% used for \Qitem and \QItem
\usepackage{typearea}
\usepackage{array}
\areaset{17cm}{26cm}
\setlength{\topmargin}{-1cm}
\usepackage{scrlayer-scrpage}
\pagestyle{scrheadings}
\ihead{Auditors' perspective on smart contracts security}
\ohead{\pagemark}
\chead{}
\cfoot{}

%%%%%%%%%%%%%%%%%%%%%%%%%%%%%%%%%%%%%%%%%%%%%%%%%%%%%%%%%%%%
%% Beginning of questionnaire command definitions %%
%%%%%%%%%%%%%%%%%%%%%%%%%%%%%%%%%%%%%%%%%%%%%%%%%%%%%%%%%%%%
%%
%% 2010, 2012 by Sven Hartenstein
%% mail@svenhartenstein.de
%% http://www.svenhartenstein.de

%% \Qq = Questionaire question. Oh, this is just too simple. It helps
%% making it easy to globally change the appearance of questions.
\newcommand{\Qq}[1]{\textbf{#1}}

%% \QO = Circle or box to be ticked. Used both by direct call and by
%% \Qrating and \Qlist.
\newcommand{\QO}{$\Box$}% or: $\ocircle$


%% \Qline = Again, this is very simple. It helps setting the line
%% thickness globally. Used both by direct call and by \Qlines.
\newcommand{\Qline}[1]{\noindent\rule{#1}{0.6pt}}

%% \Qlines = Insert NUM lines with width=\linewith. You can change the
%% \vskip value to adjust the spacing.
\newcounter{ql}
\newcommand{\Qlines}[1]{\forloop{ql}{0}{\value{ql}<#1}{\vskip0em\Qline{\linewidth}}}

%% \Qlist = This is an environment very similar to itemize but with
%% \QO in front of each list item. Useful for classical multiple
%% choice. Change leftmargin and topsep accourding to your taste.
\newenvironment{Qlist}{%
\renewcommand{\labelitemi}{\QO}
\begin{itemize}[leftmargin=1.5em,topsep=-.5em]
}{%
\end{itemize}
}



%% \Qitem = Item with automatic numbering. 
\newcounter{itemnummer}
\newcommand{\Qitem}[2][]{% #1 optional, #2 notwendig
\ifthenelse{\equal{#1}{}}{\stepcounter{itemnummer}}{}
\ifthenelse{\equal{#1}{a}}{\stepcounter{itemnummer}}{}
\begin{enumerate}[topsep=2pt,leftmargin=2.8em]
\item[\textbf{\arabic{itemnummer}#1.}] #2
\end{enumerate}
}

%% \QItem = Like \Qitem but with alternating background color. This
%% might be error prone as I hard-coded some lengths (-5.25pt and
%% -3pt)! I do not yet understand why I need them.
\definecolor{bgodd}{rgb}{0.8,0.8,0.8}
\definecolor{bgeven}{rgb}{0.9,0.9,0.9}
\newcounter{itemoddeven}
\newlength{\gb}
\newcommand{\QItem}[2][]{% #1 optional, #2 notwendig
\setlength{\gb}{\linewidth}
\addtolength{\gb}{-5.25pt}
\ifthenelse{\equal{\value{itemoddeven}}{0}}{%
\noindent\colorbox{bgeven}{\hskip-3pt\begin{minipage}{\gb}\Qitem[#1]{#2}\end{minipage}}%
\stepcounter{itemoddeven}%
}{%
\noindent\colorbox{bgodd}{\hskip-3pt\begin{minipage}{\gb}\Qitem[#1]{#2}\end{minipage}}%
\setcounter{itemoddeven}{0}%
}
}

%%%%%%%%%%%%%%%%%%%%%%%%%%%%%%%%%%%%%%%%%%%%%%%%%%%%%%%%%%%%
%% End of questionnaire command definitions %%
%%%%%%%%%%%%%%%%%%%%%%%%%%%%%%%%%%%%%%%%%%%%%%%%%%%%%%%%%%%%

\begin{document}

\begin{center}
\textbf{\huge Smart Contracts Security: How Far Are We? Auditors' perspective.}
\end{center}\vskip1em

\section{Demographics}
The following questions are intended to gauge the experience level of the participants and understand the nature of the dapps they are working on.

\QItem{ \Qq{Please specify your smart contract auditing experience in years.}
\begin{Qlist}
    \item [A] less than 1
    \item [B] 1-2
    \item [C] 3-5
    \item [D] 5+
\end{Qlist}
}


\QItem{ \Qq{What is the size of the organization you work for (the number of people involved)?}
\begin{Qlist}
    \item [A] 1-5
    \item [B] 5-25
    \item [C] 25-50
    \item [D] 50-250
    \item [E] 251+
    \item [F] I'm an independent auditor
\end{Qlist}
}

\QItem{ \Qq{What are the main blockchains where your organization's clients deploy their applications?}
\begin{Qlist}
    \item [A] Ethereum
    \item [B] BSC
    \item [C] Polygon
    \item [D] Arbitrum
    \item [E] Avalanche
    \item [F] Solana
    \item [G] Fantom
    \item [H] Other
\end{Qlist}
}


\QItem{ \Qq{Which resources do you use to get informed about smart contract/DeFi security and the latest exploits?}
Go to the next question if you do not use any resources
\begin{Qlist}
    \item [A] Research Papers
    \item [B] Social Media (e.g., Twitter, YouTube)
    \item [C] Blogposts (e.g., medium)
    \item [D] Security companies guides
    \item [E] News Websites (e.g., coindesk, rekt)
    \item [F] Colleagues
    \item [G] Questionnaire/Answer Websites (e.g., StackOverflow)
    \item [H] Conferences/Security workshops
    \item [I] Other
\end{Qlist}
}



\section{Familiarity/experience with security tools}
\QItem{ \Qq{Which of the following type of tools have you used?}
\begin{Qlist}
    \item [A] IDE (e.g., Remix)
    \item [B] Linter (e.g., EthLinter)
    \item [C] Static analysis (e.g., slither)
    \item [D] Fuzzer (e.g., Echidna)
    \item [E] Symbolic execution (e.g., Mythril)
    \item [F] Automated patcher/Bytecode hardening (e.g., sGuard)
    \item [G] Model checking/Formal verification (e.g., VeriSmart)
    \item [H] Runtime monitoring
    \item [I] Developers toolkits (e.g., hardhat/foundry)
    \item [J] Other
\end{Qlist}
}

\QItem{ \Qq{Below we enumerate a number of security analysis characteristics. To what extent do you agree regarding their importance?}

\begin{tabular}{|>{\centering\arraybackslash}p{3cm}|*{5}{>{\centering\arraybackslash}p{2cm}|}}
    \hline
& \textbf{Strongly Disagree} & \textbf{Disagree} & \textbf{Neither agree nor disagree} & \textbf{Agree} & \textbf{Strongly Agree} \\
\hline
Low false positives (tool does not wrongly indicate that there is a vulnerability) & \QO & \QO & \QO & \QO & \QO \\
\hline
Ease of use (e.g., simple CLI, integrated into IDE) & \QO & \QO & \QO & \QO & \QO \\
\hline
Documentation & \QO & \QO & \QO & \QO & \QO \\
\hline
Report quality (i.e., detail of vulnerabilities detected) & \QO & \QO & \QO & \QO & \QO \\
\hline
\end{tabular}
}

\QItem{ \Qq{What other factors could positively or negatively affect you when using a security analysis tool?}: 
\Qline{8cm} }

\QItem{ \Qq{Which vulnerabilities do you think are more \textbf{difficult} to be detected \textbf{manually}?}
\begin{Qlist}
    \item [A] Reentrancy
    \item [B] Absence of coding logic or sanity check
    \item [C] Logic errors
    \item [D] Oracle manipulation
    \item [E] Token standard incompatibility
    \item [F] Function/State Visibility Error
    \item [G] Improper asset locks or frozen asset
    \item [H] Unhandled or mishandled exception
    \item [I] Timestamp Dependence
    \item [J] Integer Overflow and Underflow
    \item [K] Other
\end{Qlist}
}

\QItem{ \Qq{Which vulnerabilities that you deem crucial \textbf{cannot} be detected by \textbf{automated security} tools that you have used?}
\begin{Qlist}
    \item [A] Reentrancy
    \item [B] Absence of coding logic or sanity check
    \item [C] Logic errors
    \item [D] Oracle manipulation
    \item [E] Token standard incompatibility
    \item [F] Function/State Visibility Error
    \item [G] Improper asset locks or frozen asset
    \item [H] Unhandled or mishandled exception
    \item [I] Timestamp Dependence
    \item [J] Integer Overflow and Underflow
    \item [K] Other
\end{Qlist}
}

\QItem{\Qq{For each of the vulnerabilities you selected in question you selected:  \textit{(Which vulnerabilities do you think are more difficult to be detected?)}, write which methods you think are the most efficient to detect it (e.g., reentrancy: manually / fuzzing).
\Qline{8cm} }}


\QItem{ \Qq{Do you find security tools helpful when auditing smart contracts?}
\hskip0.4cm \QO{} 1
\hskip0.5cm \QO{} 2
\hskip0.5cm \QO{} 3
\hskip0.5cm \QO{} 4
\hskip0.5cm \QO{} 5}

\QItem{ \Qq{If you have developed an automated security tool specify its type}
\begin{Qlist}
    \item [A] Static Analyzer
    \item [B] Fuzzer
    \item [C] Symbolic Executor
    \item [D] Formal Verification tool
    \item [E] Runtime monitor tool
\end{Qlist}
}

\section{Security practices in the current organization}
If you are an independent auditor answer the following questions on how you perform audits.


\QItem{ \Qq{What techniques does the organization you work for employ in audits?}
\begin{Qlist}
    \item [A] Manual Auditing
    \item [B] Runtime monitoring
    \item [C] Running/Extending Tests (e.g., unit testing, integration testing)
    \item [D] Using offline tools (e.g., static analyzers and fuzzers)
    \item [E] Other
\end{Qlist}
}

\QItem{ \Qq{Does the organization you work for develop in-house automated tools?}
\begin{Qlist}
    \item [A] Yes
    \item [B] Yes and it open-source some of them
    \item [C] No
\end{Qlist}
}

\QItem{ \Qq{Specify the tools that the team you work with employs.}
\begin{Qlist}
    \item [A] Slither
    \item [B] Manticore
    \item [C] Mythril
    \item [D] Echidna
    \item [E] Foundry's propert-based fuzzer
    \item [F] Remix-analyzer
    \item [G] Oyente
    \item [H] Ethlint
    \item [I] Solhint
    \item [J] Maian
    \item [K] Securify2
    \item [L] MythX (Consensys)
    \item [M] Certora’s prover
    \item [N] Other linter (similar to Ethlint)
    \item [O] Other static analyzer (similar to Slither)
    \item [P] Other fuzzer (similar to Echidna)
    \item [Q] Other symbolic execution tool (similar to Mythril)
    \item [R] Other formal verification / model checking tool (e.g., Dafny or Act)
    \item [S] Other runtime monitoring (e.g., forta)
    \item [T] Other
\end{Qlist}
}

\QItem{ \Qq{If the organization you work for runs any tool that requires specifications (e.g., for formal verification), who is going to write them?}
Go to the next question if you don't use such tools.
\begin{Qlist}
    \item [A] Developers of the dapp
    \item [B] Auditors
    \item [C] Auditors and developers
    \item [D] I don't know
    \item [E] Other
\end{Qlist}
}

\QItem{ \Qq{Does the organization you work for hire external security auditors for auditing its protocols?}
Go to the next question if you don't use such tools.
\begin{Qlist}
    \item [A] 0\%-20\%
    \item [B] 21\%-40\%
    \item [C] 41\%-60\%
    \item [D] 61\%-80\%
    \item [E] 81\%-100\%
\end{Qlist}
}

\QItem{ \Qq{What other strategies (if any) does the team you work with use to audit smart contracts?}: 
\Qline{8cm} }


\end{document}